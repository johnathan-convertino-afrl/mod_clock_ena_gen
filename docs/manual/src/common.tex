\begin{titlepage}
  \begin{center}

  {\Huge MOD\_CLOCK\_ENABLE\_GENERATOR}

  \vspace{25mm}

  \includegraphics[width=0.90\textwidth,height=\textheight,keepaspectratio]{img/AFRL.png}

  \vspace{25mm}

  \today

  \vspace{15mm}

  {\Large Jay Convertino}

  \end{center}
\end{titlepage}

\tableofcontents

\newpage

\section{Usage}

\subsection{Introduction}

\par
This core will generate an enable that is some divisible rate of the clock. The pulse will last for one clock cycle. This
core is meant to be used in situations where the enable on a register is used to control the clock rate.

\subsection{Dependencies}

\par
The following are the dependencies of the cores.

\begin{itemize}
  \item fusesoc 2.X
  \item iverilog (simulation)
  \item cocotb (simulation)
\end{itemize}

\input{src/fusesoc/depend_fusesoc_info.tex}

\subsection{In a Project}
\par
Simply use this core to generate a slow enable for a clocked device.

\section{Architecture}
\par
The only module is the mod\_clock\_ena\_gen module. It is listed below.

\begin{itemize}
  \item \textbf{mod\_clock\_ena\_gen} Implement an algorithm to generate a slow enable based on a faster clock (see core for documentation).
\end{itemize}

\par
This method of generating a slow enable allows for a design to save clocks and prevent timing issues. Since the timing is still based on the
original clock and the enable signal is sychronus to it. It can suffer from jitter and deviation, but for devices such as a UART this is well
within its tolerance. In testing this seems to be 5 percent at worst down to 0 percent at best.

\par
Method is based on a mod divide of the clock freqency and the requested output enable rate.
\begin{enumerate}
\item Add the requested enable rate to a counter
\item Set enable to 0
\item If the counter is equal to or greater than the clock frequency
  \begin{enumerate}
    \item Set the counter to counter minus clock frequency. This will result in the overflow being set to the counter for the next build up.
    \item Set enable to 1
  \end{enumerate}
\end{enumerate}

Please see \ref{Module Documentation} for more information.

\section{Building}

\par
The mod clock enable core is written in Verilog 2001. They should synthesize in any modern FPGA software. The core comes as a fusesoc packaged core and can be
included in any other core. Be sure to make sure you have meet the dependencies listed in the previous section.

\subsection{fusesoc}
\par
Fusesoc is a system for building FPGA software without relying on the internal project management of the tool. Avoiding vendor lock in to Vivado or Quartus.
These cores, when included in a project, can be easily integrated and targets created based upon the end developer needs. The core by itself is not a part of
a system and should be integrated into a fusesoc based system. Simulations are setup to use fusesoc and are a part of its targets.

\subsection{Source Files}

\subsubsection{fusesoc\_info File List}
\begin{itemize}
\item src
	\begin{itemize}
	\item src/mod\_clock\_ena\_gen.v
	\end{itemize}
\item tb
	\begin{itemize}
	\item {'tb/tb\_mod\_ena.v': {'file\_type': 'verilogSource'}}
	\end{itemize}
\item tb\_cocotb
	\begin{itemize}
	\item {'tb/tb\_cocotb.py': {'file\_type': 'user', 'copyto': '.'}}
	\item {'tb/tb\_cocotb.v': {'file\_type': 'verilogSource'}}
	\end{itemize}
\end{itemize}


\subsection{Targets}

\input{src/fusesoc/targets_fusesoc_info.tex}

\subsection{Directory Guide}

\par
Below highlights important folders from the root of the directory.

\begin{enumerate}
  \item \textbf{docs} Contains all documentation related to this project.
    \begin{itemize}
      \item \textbf{manual} Contains user manual and github page that are generated from the latex sources.
    \end{itemize}
  \item \textbf{src} Contains source files for the core
  \item \textbf{tb} Contains test bench files for iverilog and cocotb
\end{enumerate}

\newpage

\section{Simulation}
\par
There are a few different simulations that can be run for this core. All currently use iVerilog (icarus) to run. The first is iverilog, which
uses verilog only for the simulations. The other is cocotb. This does a unit test approach to the testing and gives a list of tests that pass
or fail.

\subsection{iverilog}
\par
All simulation targets that do NOT have cocotb in the name use a verilog test bench with verilog stimulus components.
All of these tests provide fst output files for viewing the waveform in the there target build folder.

\subsection{cocotb}
\par

To use the cocotb tests you must install the following python libraries.

\begin{lstlisting}[language=bash]
 $ pip install cocotb
\end{lstlisting}

Then you must use the cocotb sim target. In this case it is sim\_cocotb. This target can be run with various bus and fifo parameters.

\begin{lstlisting}[language=bash]
  $ fusesoc run --target sim_cocotb AFRL:clock:mod_clock_ena_gen:1.0.0
\end{lstlisting}

\newpage

\section{Code Documentation} \label{Code Documentation}

\par
Natural docs is used to generate documentation for this project. The next lists the following sections.

\begin{itemize}
\item \textbf{mod\_clock\_ena\_gen} Generate a low rate enable clock.\\
\item \textbf{tb\_mod\_ena} Verilog test bench.\\
\item \textbf{tb\_cocotb verilog} Verilog test bench base for cocotb.\\
\item \textbf{tb\_cocotb python} cocotb unit test functions.\\
\end{itemize}

